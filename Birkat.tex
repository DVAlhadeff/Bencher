\documentclass[a5paper, 12pt]{Birchon}

\title{Birkat}

\begin{document}

\fancyhead{} % clear all header fields
\cfoot{\thepage}

\chead{\fontsize{10}{12} \selectfont   ברכת המזון {\textendash} Introduction }

\begin{english}
{\tableofcontents}
\end{english}

\eject

\begin{english}
\addcontentsline{toc}{section}{Foreword}
\section*{Foreword}
Short `letter' from ``Dan and Zan''

\subsection*{How is this bentcher birkon different from all the others?}

\addcontentsline{toc}{section}{Introduction}
\section*{Introduction}
\subsection*{Why do we bless \emph{after} meals?}
\subsection*{What is Havdalah?}
\end{english}

\eject

\fancyhead[C]{\fontsize{10}{12} \selectfont   ברכת המזון {\textendash} Meals After Grace }

\begin{english}
\addcontentsline{toc}{section}{Grace After Meals}
\section*{Grace After Meals}
\subsection*{Introduction to Grace After Meals / Hymns of Praise}

\begin{tabu} to 1\textwidth { X[.2,r] X[l] X[l] }
\begin{hebrew}
אין
\end{hebrew}
&
Non como muestro Dio, \newline Non como muestro señor,\newline Non como muestro rey, \newline Non como muestro salvador!
\vskip .5em
&
There is none like our G--d, \newline There is none like our LORD,\newline There is none like our King, \newline There is none like our Savior!
\vskip .5em
\\
\begin{hebrew}
מי 
\end{hebrew}
&
Quien como muestro Dio, \newline Quien como muestro señor, \newline Quien como muestro rey, \newline Quien como muestro salvador?
\vskip .5em
&
Who is like our G--d, \newline Who is like our LORD,\newline Who is like our King, \newline Who is none like our Savior?
\vskip .5em
\\
\begin{hebrew}
נודה 
\end{hebrew}
&
Loaremos a muestro Dio, \newline Loaremos a muestro Señor,\newline Loaremos a muestro Rey, \newline Loaremos a muestro Salvador!
\vskip .5em
&
Give thanks to our G--d, \newline Give thanks to our LORD,\newline Give thanks to our King, \newline Give thanks to our Savior?
\vskip .5em
\\
\begin{hebrew}
ברוך  
\end{hebrew}
&
Bendicho muestro Dio, \newline Bendicho muestro Señor, \newline Bendicho muestro Rey, \newline Bendicho muestro Salvador!
\vskip .5em
&
Blessed is our G--d, \newline Blessed is our LORD,\newline Blessed is our King, \newline Blessed is our Savior?
\vskip .5em
\\
\begin{hebrew}
אתה 
\end{hebrew}
&
Tu sos muestro Dio,\newline Tu sos muestro Señor,\newline Tu sos muestro Rey,\newline Tu sos muestro Salvador.
\vskip .5em
&
You are our G--d, \newline You are our LORD,\newline You are our King, \newline You are our Savior?
\vskip .5em
\end{tabu}
\end{english}

\eject

\begin{english}
\addcontentsline{toc}{subsection}{Invitation to Say Grace / Zimun}
\subsection*{Invitation to Say Grace}
\end{english}

\regPage{
}
{}
{
Talmud Berahot 48b states:
\begin{quote}
\end{quote}

}

\eject

\regPage{
\begin{english}
\addcontentsline{toc}{subsection}{Blessing of Sustenance}
\subsection*{Blessing of Sustenance}
\end{english}

\lettrine[lines=2, depth=1, findent=1.5em]{בָּרוּךְ} אַתָּה יְהֹוָה אֱלֹהֵינוּ מֶלֶךְ הָעוֹלָם, הָאֵל הַזָּן אוֹתָנוּ וְאֶת הָעוֹלָם כֻּלּוֹ בְּטוּבוֹ, בְּחֵן בְּחֶסֶד בְּרֵיוַח וּבְרַחֲמִים רַבִּים, נֹתֵן לֶחֶם לְכָל בָּשָׂר. כִּי לְעוֹלָם חַסְדּוֹ: וּבְטוּבוֹ הַגָּדוֹל, תָּמִיד לֹא חָסַר לָנוּ, וְאַל יֶחְסַר לָנוּ מָזוֹן תָּמִיד לְעוֹלָם וָעֶד. כִּי הוּא אֵל זָן וּמְפַרְנֵס לַכֹּל, וְשֻׁלְחָנוֹ עָרוּךְ לַכֹּל, וְהִתְקִין מִחְיָה וּמָזוֹן לְכָל בְּרִיּוֹתָיו אֲשֶׁר בָּרָא בְרַחֲמָיו וּבְרוֹב חֲסָדָיו, כָּאָמוּר.פּוֹתֵחַ אֶת יָדֶךָ. וּמַשְׂבִּיעַ לְכָל חַי רָצוֹן: בָּרוּךְ אַתָּה יְהֹוָה הַזָּן אֶת הַכֹּל:
}
{
\lettrine[]{B}lessed are You, L--rd our G--d, King of the universe, Who, in His goodness, provides sustenance for the entire world with grace, with kindness, and with mercy. He gives food to all flesh, for His kindness is everlasting. Through His great goodness to us continuously we do not lack [food], and may we never lack food, for the sake of His great Name. For He, benevolent G‑d, provides nourishment and sustenance for all, does good to all, and prepares food for all His creatures whom He has created, as it is said: You open Your hand and satisfy the desire of every living thing. Blessed are You, L-rd, Who provides food for all.
}
{
The first blessing is about the LORD providing sustenance to All his creations. It's pretty short and sweet, and it is customary to sing in a pleasing tune on Sabbaths and Festivals when eating with a group. The Spanish Portuguese add several words to the beginning of the blessing (as compared to the Ashkenaz or Edot Mizrach text): ``who\ldots''
}

\eject

\regPage{
\begin{english}
Barukh atah Adonai, elohenu melekh ha'olam, hazan et ha'olam kulo betuvo
\end{english}
}
{

}
{
}

\eject

\begin{english}
\addcontentsline{toc}{subsection}{Blessing of the Land}
\subsection*{Blessing of the Land}
\end{english}

\eject

\begin{english}
\addcontentsline{toc}{subsection}{Blessing of Jerusalem}
\subsection*{Blessing of Jerusalem}
\end{english}

\eject

\begin{english}
\addcontentsline{toc}{subsection}{Blessings of Well-being}
\subsection*{Blessings of Well-being}
\end{english}

\eject

\begin{english}
\addcontentsline{toc}{subsection}{Concluding Hymns}
\subsection*{Concluding Hymns}

\subsubsection*{Bendigamos}

\begin{tabu} to 1\textwidth { X[l] X[l] }
Bendigamos al Altísimo, \newline 
al Señor que nos crió,
Démosle agradecimiento
Por los bienes que nos dió.
\vskip .5em
&
Let us bless the Most High,
The Lord who created us,
Let us give him thanks
For the good things he has given us.
\vskip .5em
\\
Alabado sea su Santo Nombre,
Porque siempre nos apiadó.
Odu L'adonai ki tov,
Ki leolam jasdó.
\vskip .5em
&
Praised be his Holy Name,
For he has always taken pity on us.
Praise the Lord, for he is good,
For his mercy is everlasting.
\vskip .5em
\\
Bendigamos al Altísimo, \newline 
Por su Ley primeramente,
Que liga a nuestra casa
Con el cielo continuamente,

{\textit{Alabado sea}\ldots}
\vskip .5em
&
Let us bless the Most High,
First, for his Law,
Which connects our home,
With heaven, continuously.
\vskip .5em
\\
Bendigamos al Altísimo, \newline 
Por el pan segundamente,
Y también por los manjares
Que comimos juntamente.

{\textit{Alabado sea}\ldots}
\vskip .5em
&
Let us bless the Most High,
Second, for the bread
And also for these foods
Which we have eaten together.
\vskip .5em
\\
Pues comimos y bebimos alegremente
Su merced nunca nos faltó.
Odu L'adonai ki tov,
Ki leolam jasdó.
\vskip .5em
&
For we have eaten and drunk happily,
His mercy has never failed us.
Praise the Lord, for he is good,
For his mercy is everlasting.
\vskip .5em
\\
Bendita sea la casa esta -- \newline el hogar de su presencia,
donde guardamos su fiesta,
con alegría y permanencia.

{\textit{Alabado sea}\ldots}
&
Blessed be this house -- \newline the home of His presence,
Where we keep his feast, with happiness and steadfastness.
\\
\end{tabu}
\end{english}

\eject

\begin{english}
\addcontentsline{toc}{subsection}{The Seven Blessings of Marriage}
\subsection*{The Seven Blessings of Marriage}
\end{english}

\eject

\begin{english}
\addcontentsline{toc}{section}{Friday Night/Festival Kiddush}
\section*{Friday Night/Festival Kiddush}
\end{english}

\eject

\begin{english}
\addcontentsline{toc}{section}{Havdalah}
\section*{Havdalah}
\end{english}

\eject

\begin{english}
\addcontentsline{toc}{section}{Table Songs}
\section*{Table Songs}

\addcontentsline{toc}{subsubsection}{Purim lano,\ldots Pesah a la mano}
\subsubsection*{Purim lano,\ldots Pesah a la mano}

\begin{tabu} to 1\textwidth { X[l] X[l] }
Purim, Purim, Purim lano \newline
Pesah, Pesah a la mano \newline
Las masas si stan faziendo, los japrakis si stan koziendo
\vskip .5em
&
Purim is over, Pesach is at hand \newline
Matzot are being made, stuffed leaves are being baked.
\vskip .5em
\\
Aman, aman, aman, aman
Il Dio bendicho mos da mazal
\vskip .5em
&
Aman\ldots
The LORD gives us good fortune.
\vskip .5em
\\
Purim, Purim, Purim lano \newline
Pesah, Pesah a la mano \newline
La Nona sta diziendo a los nyetas
Alimpia il puevlo, kontones i los techos \newline
{\textit{Aman, aman\ldots}}
\vskip .5em
&
Purim is over, Pesach is at hand \newline
The grandmother tells the grandchildren, ``Clean the dust -- corners and ceilings''
\vskip .5em
\\
Purim, Purim, Purim lano \newline
Pesah, Pesah a la mano \newline
Il senor Rubi disho a las tiyas
No kumer il pan ocho dias \newline
{\textit{Aman, aman\ldots}}
\vskip .5em
&
Purim is over, Pesach is at hand \newline
The Rabbi tells the aunts, ``No eating bread for eight days''.
\end{tabu}

\clearpage

\addcontentsline{toc}{subsubsection}{Cuando El Rey Nimrod}
\subsubsection*{Cuando El Rey Nimrod}

\begin{tabu} to 1\textwidth { X[l] X[l] }
Cuando il rey Nimrod al campo salía,
Mirava en el cielo y en la estreyeria,
Vido una luz santa en la djuderiya,
Que havía de nacer Avraham avinu.
\vskip .5em
&
When King Nimrod went out the fields
And looked among the stars,
He saw a light in the Jewish quarter,
Foretelling that Abraham our father would be born.
\vskip .5em
\\
Avram avinu, padre querido,
Padre bedicho, luz de Israel.
\vskip .5em
&
Abraham, our father, beloved father, light of Israel\ldots x2
\\
La mujer de Terah quedo preñada,
De día en día el la preguntava:
–De qué teneij la cara tan demudada?
Ella ya savía el bien
que tenía. {\textit{Avram\ldots}}
\vskip .5em
&
Terach’s wife was pregnant
And each day he would ask her,
“Why do you look so distraught?”
She already knew very well what she had.
\vskip .5em
\\
En fin de nueve mézes
parir quería,
Iva caminando por campos y viñas,
A su marido tal no le descurvría.
Topó una meara ayi la pariría.
{\textit{Avram\ldots}}
\vskip .5em
&
At the end of nine months she was determined to give birth,
She walked through the fields and vineyards
She didn’t tell her husband anything,
She found a manger [cave]; there, she would give birth.
\vskip .5em
\end{tabu}

\clearpage

\addcontentsline{toc}{subsubsection}{Scalerica de Oro}
\subsubsection*{Scalerica de Oro}
\begin{tabu} to 1\textwidth { X[l] X[l] }
Scalerica de oro, de oro i de marfíl,
Para que suva la novia a dar kidushim.
&
A little stairway of gold, of gold and ivory;
the bride will ascend the stairway to take her wedding vows.
\vskip .5em
\\
Venimos a ver, venimos a ver; \newline
I gozen i logren i tengan mucho bien.
\vskip .5em
&
We will come to see, we will come to see. \newline
May they have joy and prosperity
And many good things.
\vskip .5em
\\
La novia no tiene dinero,
Quémos tengan un mazal bueno.
\vskip .5em
&
The bride has no dowry;
May she have good fortune.
\vskip .5em
\\
La novia no tiene contado,
La novia no tiene dinero,
Quémos tengan un mazal alto.
\vskip .5em
&
The bride has no riches,
The bride has no dowry;
May she have the best of luck.
\vskip .5em
\end{tabu}

\end{english}

\end{document}