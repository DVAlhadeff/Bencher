\documentclass[a5paper, 12pt]{Birchon}

\title{Birkat}

\begin{document}

\fancyhead{} % clear all header fields
\fancyfoot[C]{\thepage}
\fancyfoot[R]{}

\fancyhead[C]{\fontsize{10}{12} \selectfont   ברכת המזון {\textendash} Introduction }

\begin{english}
{\tableofcontents}
\end{english}

\eject

\begin{english}
\addcontentsline{toc}{section}{Foreword}
\section*{Foreword}
Short `letter' from ``Dan and Zan''

\subsection*{How is this bentcher birkon different from all the others?}

\addcontentsline{toc}{section}{Introduction}
\section*{Introduction}
\subsection*{Why do we bless \emph{after} meals?}
\subsection*{What is Havdalah?}
\end{english}

\eject

\fancyhead[C]{\fontsize{10}{12} \selectfont   ברכת המזון {\textendash} Meals After Grace }

\begin{english}
\addcontentsline{toc}{section}{Grace After Meals}
\section*{Grace After Meals}
\subsection*{Introduction to Grace After Meals / Hymns of Praise}
\begin{verse}

Non como muestro Dio,\\Non como muestro señor,\\
Non como muestro rey,\\Non como muestro salvador.

Quien como muestro Dio,\\Quien como muestro señor,\\
Quien como muestro rey,\\Quien como muestro salvador.

Loaremos a muestro Dio,\\Loaremos a muestro señor,\\
Loaremos a muestro rey,\\Loaremos a muestro salvador.

Bendicho muestro Dio,\\Bendicho muestro señor,\\
bendicho muestro rey,\\Bendicho muestro salvador.

Tu sos muestro Dio,\\Tu sos muestro señor,\\
Tu sos muestro rey,\\Tu sos muestro salvador.

\end{verse}
\end{english}

\eject

\begin{english}
\addcontentsline{toc}{subsection}{Invitation to Say Grace / Zimun}
\subsection*{Invitation to Say Grace}
\end{english}

\regPage{
}
{}
{
Talmud Berahot 48b states:
\begin{quote}
\end{quote}

}

\eject

\regPage{
\begin{english}
\addcontentsline{toc}{subsection}{Blessing of Sustenance}
\subsection*{Blessing of Sustenance}
\end{english}

\lettrine[lines=2, depth=1, findent=1.5em]{בָּרוּךְ} אַתָּה יְהֹוָה אֱלֹהֵינוּ מֶלֶךְ הָעוֹלָם, הָאֵל הַזָּן אוֹתָנוּ וְאֶת הָעוֹלָם כֻּלּוֹ בְּטוּבוֹ, בְּחֵן בְּחֶסֶד בְּרֵיוַח וּבְרַחֲמִים רַבִּים, נֹתֵן לֶחֶם לְכָל בָּשָׂר. כִּי לְעוֹלָם חַסְדּוֹ: וּבְטוּבוֹ הַגָּדוֹל, תָּמִיד לֹא חָסַר לָנוּ, וְאַל יֶחְסַר לָנוּ מָזוֹן תָּמִיד לְעוֹלָם וָעֶד. כִּי הוּא אֵל זָן וּמְפַרְנֵס לַכֹּל, וְשֻׁלְחָנוֹ עָרוּךְ לַכֹּל, וְהִתְקִין מִחְיָה וּמָזוֹן לְכָל בְּרִיּוֹתָיו אֲשֶׁר בָּרָא בְרַחֲמָיו וּבְרוֹב חֲסָדָיו, כָּאָמוּר.פּוֹתֵחַ אֶת יָדֶךָ. וּמַשְׂבִּיעַ לְכָל חַי רָצוֹן: בָּרוּךְ אַתָּה יְהֹוָה הַזָּן אֶת הַכֹּל:
}
{
\lettrine[]{B}lessed are You, L--rd our G--d, King of the universe, Who, in His goodness, provides sustenance for the entire world with grace, with kindness, and with mercy. He gives food to all flesh, for His kindness is everlasting. Through His great goodness to us continuously we do not lack [food], and may we never lack food, for the sake of His great Name. For He, benevolent G‑d, provides nourishment and sustenance for all, does good to all, and prepares food for all His creatures whom He has created, as it is said: You open Your hand and satisfy the desire of every living thing. Blessed are You, L-rd, Who provides food for all.
}
{
The first blessing is about the LORD providing sustenance to All his creations. It's pretty short and sweet, and it is customary to sing in a pleasing tune on Sabbaths and Festivals when eating with a group. The Spanish Portuguese add several words to the beginning of the blessing (as compared to the Ashkenaz or Edot Mizrach text): ``who\ldots''
}

\eject

\regPage{
\begin{english}
Barukh atah Adonai, elohenu melekh ha'olam, hazan et ha'olam kulo betuvo
\end{english}
}
{

}
{
}

\begin{english}
\addcontentsline{toc}{subsection}{Blessing of the Land}
\subsection*{Blessing of the Land}
\end{english}

\begin{english}
\addcontentsline{toc}{subsection}{Blessing of Jerusalem}
\subsection*{Blessing of Jerusalem}
\end{english}

\begin{english}
\addcontentsline{toc}{subsection}{Blessings of Well-being}
\subsection*{Blessings of Well-being}
\end{english}

\begin{english}
\addcontentsline{toc}{subsection}{Concluding Hymns}
\subsection*{Concluding Hymns}
\end{english}

\eject

\begin{english}
\addcontentsline{toc}{subsection}{The Seven Blessings of Marriage}
\subsection*{The Seven Blessings of Marriage}
\end{english}

\eject

\begin{english}
\addcontentsline{toc}{section}{Friday Night/Festival Kiddush}
\section*{Friday Night/Festival Kiddush}
\end{english}

\eject

\begin{english}
\addcontentsline{toc}{section}{Havdalah}
\section*{Havdalah}
\end{english}

\eject

\begin{english}
\addcontentsline{toc}{section}{Table Songs}
\section*{Table Songs}

\addcontentsline{toc}{subsubsection}{Scalerica de Oro}
\subsubsection*{Scalerica de Oro}
\begin{Parallel}{.45\textwidth}{.45\textwidth}
\ParallelLText{
Scalerica de oro,\\
De oro y de marfíl,
Para que suva la novia
A dar kidushim.
}
\ParallelRText{
A little stairway of gold,
Of gold and ivory;
The bride will ascend the stairway
To take her wedding vows.
}
\ParallelPar
\ParallelLText{
Venimos a ver,
Venimos a ver;\\
Y gozen y logren
Y tengan mucho bien.
}
\ParallelRText{
We will come to see,
We will come to see.\\
May they have joy and prosperity
And many good things.
}
\ParallelPar
\ParallelLText{
La novia no tiene dinero,
Quémos tengan un mazal bueno.
}
\ParallelRText{
The bride has no dowry;
May she have good fortune.
}
\ParallelPar
\ParallelLText{
La novia no tiene contado,
La novia no tiene dinero,
Quémos tengan un mazal alto.
}
\ParallelRText{
The bride has no riches,
The bride has no dowry;
May she have the best of luck.
}
\end{Parallel}

\end{english}

\end{document}